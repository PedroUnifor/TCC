\chapter{Resultados}

\section{PRIMEIRO TESTE}
\label{primeiro_teste}
O primeiro teste do projeto foi realizado na pista de atletismo da Unifor, no dia 8 de dezembro.
Logo ao chegar na pista de atletismo deparamos com o primeiro problema: o cabo de força não era grande o suficiente para atingir todo o percurso que escolhemos para treinar o Jaguar. Pedimos então aos auxiliares por uma nova extensão para o cabo que foi entregue em pouco tempo. O jaguar foi treinado em apenas uma parte da pista, indo e voltando duas vezes. o caminho percorrido é mostrado na figura \ref{pistaunifor}.

	\begin{figure}[H]
		\centering
		\Caption{\label{pistaunifor} Vista Aérea da Pista da Unifor}
		\UNIFORfig{}{
			\fbox{\includegraphics[width=16cm]{figuras/PistaUnifor.png}}
		}{
			\Fonte{\cite{pistaunifor}}
		}	
\end{figure}

Foram obtidas 977 imagens em um percurso de ida e volta na pista em diferentes faixas, com o intuito de gravar o imagens com as diferentes cores da pista. Elas foram obtidas pela câmera do jaguar usando o programa \textit{savefile.py} enquanto um aluno pilotava o robô. As imagens são tiradas a cada movimento novo realizado no \textit{Joystick}. Na imagem \ref{jaguarelog} podemos ver a esquerda um exemplo de imagem capturada pela câmera do jaguar e a direta o resultado do terminal de execução do programa \textit{savefile.py}.

	\begin{figure}[H]
		\centering
		\Caption{\label{jaguarelog}Exemplo de Imagem capturada pelo câmera do Jaguar e o Terminal do programa \textit{Savefile.py}}
		\UNIFORfig{}{
			\fbox{\includegraphics[width=16cm]{figuras/JaguarELog.png}}
		}{
			\Fonte{Elaborada pelo autor}
		}	
\end{figure}

O programa python tem acesso as imagens do Jaguar pelo IP de sua câmera. Ao final da captura das imagens, foi executado o programa \textit{model.py} que trata de criar o modelo de testes pro Jaguar, resultado no arquivo \textit{model-XXX.h5}, onde o "XXX" é o número do modelo da vez.

O programa \textit{model.py} tem o diretório com acesso a todas as imagens capturadas e o \textit{driver\_log} criado, caso não tenha sido mudado de lugar. Mas antes de criar o modelo, é feito um tratamento na imagem, com o auxílio do arquivo \textit{util.py}, para que seja processada com mais facilidade pela rede neural. O primeiro processo consiste em cortar o céu da imagem, que não será útil para o tratamento pela rede neural(função \textit{crop()}, letra 'B'); reduzir o tamanho da imagem, o suficiente para ainda sim conseguir ser tratada pela rede neural e não perder nenhuma característica importante (função \textit{resize()}, letra 'C'); por último, muda as cores da imagem de rgb (o tradicional formato de imagem colorida) para yuv (um formato mais simples de imagem utilizando menos cores e de processamento mais fácil) (função \textit{rgb2yuv()}, letra 'D'). Os três processos são mostrados na figura \ref{combo}, onde a primeira imagem (letra 'A') é o exemplo utilizado para realizar o processo.

	\begin{figure}[H]
		\centering
		\Caption{\label{combo} Tratamento da Imagem}
		\UNIFORfig{}{
			\fbox{\includegraphics[width=10cm]{figuras/combo.jpg}}
		}{
			\Fonte{Elaborada pelo autor}
		}	
\end{figure}

Depois da imagem tratada, é feito a criação dos modelos de rede neural pelo arquivo \textit{model.py}. Na tela de execução do programa é mostrado o tempo de criação de cada modelo, o \textit{loss} e o \textit{val\_loss},no qual o melhor será aquela com o menor \textit{val\_loss}. Para isso, é preciso modificar alguns parâmetros dentro do \textit{model.py}: \textit{np\_epoch} (defini o número de épocas de criação do modelo, ou seja, quantas vezes ele será treinado com aqueles padrões), \textit{sample\_per\_epoch} (quantidade de imagens por época) e \textit{batch\_size} (tamanho do lote de imagem de treinamento por vez). Normalmente a criação do modelo pode levar horas e exigir muito processamento do computador. Os parâmetros utilizado para cada treinamento estão na tabela \ref{parametros1} e eles foram escolhidos por serem próximos aos valores do projeto do Naoki Shibuya \cite{naokish}.

\begin{table}[H]
\centering
\caption{ Tabela de parâmetros do primeiro para primeira pista}
\label{parametros1}
\begin{tabular}{|l|l|l|l|}
\hline
\textbf{Treinamento} & \textbf{nb\_epoch} & \textbf{samples\_per\_epoch} & \textbf{batch\_size} \\ \hline
1º        & 5         & 5000                & 100         \\ 
2º        & 10        & 5000                & 100         \\ 
3º        & 5         & 10000               & 100         \\ 
4º        & 10        & 10000               & 100         \\ 
5º        & 15        & 10000               & 100         \\ 
6º        & 15        & 20000               & 150         \\ \hline
\end{tabular}
\end{table}

O \textit{model.py} está programado para criar o próximo modelo somente se houver um erro menor que o modelo anterior. Ás vezes ele cria somente um modelo pois o primeiro foi o melhor de todos, outras vezes ele cria vários modelos. Isso depende de cada caso.
No sábado seguinte, foi colocado para testar o último modelo criado, o \textit{model-009.h5} e, para a surpresa de todos, ele funcionou perfeitamente, sem a necessidade de testar nenhum outro modelo. Segue abaixo o valor do \textit{loss}:

\begin{lstlisting}
loss: 0.0937 - val\_loss: 0.0564
\end{lstlisting}

Com o notebook conectado ao Jaguar, o Programa \textit{drive.py} foi executado juntamente com o \textit{model-009.h5} e, com apenas esse primeiro modelo, foi possível colocar a plataforma para percorrer a pista de atletismo nos dois sentido. A figura \ref{testejaguar} Mostra o teste realizado na pista no qual o notebook está no chão, ao lado do robô.

	\begin{figure}[H]
		\centering
		\Caption{\label{testejaguar}Teste do Jaguar}
		\UNIFORfig{}{
			\fbox{\includegraphics[height=14cm]{figuras/testejaguar.png}}
		}{
			\Fonte{Elaborado pelo autor}
		}	
\end{figure}


Em alguns momentos o Jaguar foi deslocado um pouco da sua rota. Mesmo assim ele consegue retornar a pista. Os resultados foram impressionantes, principalmente pelo fato das inúmeras limitações para o desenvolvimento do projeto.

\section{SEGUNDO TESTE}
\label{segundo}

Para validação do trabalho, é preciso que o Jaguar consiga treinar e percorrer em outros circuitos. O segundo teste foi feita na sala L4 com o formato de pista representado na figura \ref{circuito}:

	\begin{figure}[H]
		\centering
		\Caption{\label{circuito} Circuito da pista 2}
		\UNIFORfig{}{
			\fbox{\includegraphics[width=14cm]{figuras/pistaL4.png}}
		}{
			\Fonte{Elaborado pelo autor}
		}	
\end{figure}

O segundo teste foi o mais trabalhoso. Ele foi realizado no laboratório L4 e a pista teve que ser refeita três vezes. A primeira foi a mais simples, feita de fita adesiva. Ela é mostrada na figura \ref{pista1}.

	\begin{figure}[H]
		\centering
		\Caption{\label{pista1} Primeira Pista da L4}
		\UNIFORfig{}{
			\fbox{\includegraphics[width=15cm]{figuras/pista1.jpg}}
		}{
			\Fonte{Elaborado pelo autor}
		}	
\end{figure}

O primeiro grande desafio enfrentado foi desenhar um modelo de pista no qual o Jaguar conseguisse visualizar. A fita adesiva, muito fina, não se destacava o suficiente no chão de cinza da L4. Mesmo com várias tentativas e vários modelos criados o Jaguar conseguia no máximo seguir por uma reta ou uma curva simples, mas logo se perdia. A imagem \ref{imagem1} mostra um exemplo de imagem capturada pelo Jaguar nessa pista.

	\begin{figure}[H]
		\centering
		\Caption{\label{imagem1} Imagem Capturada pela câmera do Jaguar}
		\UNIFORfig{}{
			\fbox{\includegraphics[width=14cm]{figuras/imagem1242.jpg}}
		}{
			\Fonte{Elaborado pelo autor}
		}	
\end{figure}


Mediante o problema de largura da pista, foi criado uma nova seguindo o mesmo modelo da anterior, porém com as bordas mais largas e feita de fita adesiva e papel toalha. Segue a pista na Figura \ref{pista2}:

	\begin{figure}[H]
		\centering
		\Caption{\label{pista2} Segunda Pista da L4}
		\UNIFORfig{}{
			\fbox{\includegraphics[width=15cm]{figuras/pis2.jpg}}
		}{
			\Fonte{Elaborado pelo autor}
		}	
\end{figure}


O grande problema da segunda pista foi o papel toalha. Toda vez que Jaguar errava alguma movimento e passava por cima da pista ele rasgava o papel toalha. Durante os testes e treinamento ela teve que ser refeita inúmeras vezes, mas pelo menos a segunda pista foi melhor para a visualização do Jaguar. 

	\begin{figure}[H]
		\centering
		\Caption{\label{imagem2} Imagem Capturada pela câmera do Jaguar}
		\UNIFORfig{}{
			\fbox{\includegraphics[width=14cm]{figuras/pista2imagemboa.jpg}}
		}{
			\Fonte{Elaborado pelo autor}
		}	
\end{figure}

A última pista foi feita do mesmo material da segunda, porém com a adição de fita zebrada amarela e preta nas bordas da pista. Como o algoritmo visualiza melhor com uma grande diferença de cores, a fita deu um destaque na pista, facilitando para o processamento. A figura \ref{pista3} mostra a aplicação da fita zebrada nas bordas da pista e a figura \ref{imagem3} mostra a mesma pista sendo visualizada pelo câmera do Jaguar.

	\begin{figure}[H]
		\centering
		\Caption{\label{pista3} Última Pista da L4}
		\UNIFORfig{}{
			\fbox{\includegraphics[width=15cm]{figuras/pista3.jpg}}
		}{
			\Fonte{Elaborado pelo autor}
		}	
\end{figure}

	\begin{figure}[H]
		\centering
		\Caption{\label{imagem3} Imagem Capturada pela câmera do Jaguar}
		\UNIFORfig{}{
			\fbox{\includegraphics[width=15cm]{figuras/pista3imagemruim.jpg}}
		}{
			\Fonte{Elaborado pelo autor}
		}	
\end{figure}

Para o último caso, a rede neural foi configurada com os parâmetros da tabela \ref{tabela2}:

\begin{table}[H]
\centering
\caption{Parâmetros de treinamento do segundo modelo}
\label{tabela2}
\begin{tabular}{|l|l|l|l|}
\hline
\textbf{Treinamento} & \textbf{nb\_epoch} & \textbf{samples\_per\_epoch} & \textbf{batch\_size} \\ \hline
1º        & 20        & 2000                & 200         \\
2º        & 20        & 5000                & 150         \\
3º        & 20        & 10000               & 150         \\
4º        & 30        & 20000               & 150         \\ \hline
\end{tabular}
\end{table}

Como o número de \textit{epoch} chega ao máximo de 30, poderá haver 30 modelos diferentes, mas não foi isso o que aconteceu. Como o programa está configurado para criar um modelo somente se ele tiver o \textit{val\_loss} menor que o anterior, nem sempre haverá a mesma quantidade de modelos para a quantidade de \textit{epoch} configurada. O programa fez um treinamento para cada \textit{epoch}, mas não gerou um modelo para cada um. A tabela \ref{tabela3} mostra o resultado dos cinco melhores modelos de cada treinamento realizado da Tabela \ref{tabela2}. Nela, é possível ver o número do treinamento que saiu o modelo (Coluna treinamento), a época do modelo ou número do modelo (coluna Epoch), o tempo que levou para gerar o modelo em segundos (Coluna Tempo de Prcs.(s)), o valor do \textit{Loss} (coluna \textit{Loss}) e o valor do \textit{Val\_loss} (Coluna \textit{Val\_loss}) que é o valor mais importante pois com ele é possível saber o quanto o modelo errou na sua simulação de treinamento. Quanto menor esse valor melhor poderá ser o modelo, mas isso só pode ser testado na prática. A tabela \ref{tabela3} está organizada do menor valor de \textit{Val\_loss} (no caso, o melhor modelo) para o maior. Nos testes, os modelos 8 e 15 da tabela \ref{tabela3} tiveram os melhores resultados, como é possível ver no vídeo.



\begin{table}[H]
\centering
\caption{Melhores resultados de cada treinamento Pista 2}
\label{tabela3}
\begin{tabular}{|l|l|l|l|l|l|}
\hline
\textbf{Modelo} & \textbf{Treinamento} & \textbf{Epoch} & \textbf{Tempo de Prcs.(s)} & \textit{\textbf{Loss}} & \textit{\textbf{Val\_loss}} \\ \hline
1                   & 2º                   & 11             & 114                        & 0,0716                 & 0,031                       \\
2                   & 3º                   & 4              & 209                        & 0,0697                 & 0,033                       \\
3                   & 2º                   & 6              & 108                        & 0,0725                 & 0,035                       \\
4                   & 1º                   & 17             & 46                         & 0,0735                 & 0,0355                      \\
5                   & 2º                   & 16             & 112                        & 0,0758                 & 0,0355                      \\
6                   & 3º                   & 11             & 216                        & 0,0795                 & 0,0357                      \\
7                   & 2º                   & 10             & 129                        & 0,0687                 & 0,0358                      \\
8                   & 1º                   & 13             & 47                         & 0,0723                 & 0,0368                      \\
9                   & 2º                   & 12             & 114                        & 0,0693                 & 0,0371                      \\
10                   & 3º                   & 5              & 212                        & 0,0707                 & 0,0371                      \\
11                   & 1º                   & 12             & 46                         & 0,0734                 & 0,0385                      \\
12                   & 4º                   & 3              & 416                        & 0,0686                 & 0,0389                      \\
13                   & 1º                   & 19             & 46                         & 0,0725                 & 0,039                       \\
14                   & 3º                   & 3              & 211                        & 0,0686                 & 0,0392                      \\
15                   & 3º                   & 1              & 236                        & 0,0786                 & 0,0396                      \\
16                   & 1º                   & 10             & 46                         & 0,0726                 & 0,0401                      \\
17                   & 4º                   & 2              & 420                        & 0,067                  & 0,045                       \\
18                   & 4º                   & 4              & 421                        & 0,0705                 & 0,0458                      \\
19                   & 4º                   & 1              & 423                        & 0,0736                 & 0,0473                      \\
20                   & 4º                   & 6              & 418                        & 0,0751                 & 0,0483 \\ \hline                    
\end{tabular}
\end{table}

Em comparação com os modelos da primeira pista na sessão \ref{primeiro_teste}, o resultado do \textit{val\_loss} da segunda pista foi bem menor, o que, em teoria, era para ser o modelo melhor. Na prática, o Jaguar teve um melhor desempenho da quadra de atletismo da Unifor. Na segunda pista, localizada na sala da L4, o Jaguar enfrentou alguns problemas devido o formato da pista, com curvas muito acentuadas; a cor da parede, que era a mesma cor da borda pista; e o chão, que é da mesma cor tanto dentro quanto fora da pista. Mesmo com a função \textit{crop()} do \textit{utils.py} ainda é fica visível uma pequena parte da parede, como é possível ver na figura \ref{parede}. Tudo isso pode ter gerado uma confusão no algoritmo que gera o modelo de predição da pista. 

	\begin{figure}[H]
		\centering
		\Caption{\label{parede} Exemplo Imagem cortada da Pista 2}
		\UNIFORfig{}{
			\fbox{\includegraphics[width=14cm]{figuras/parede.jpg}}
		}{
			\Fonte{Elaborado pelo autor}
		}	
\end{figure}

Ao final do Projeto, O Jaguar conseguiu percorrer a pista de atletismo totalmente autônomo perfeitamente. Na segunda pista, localizada na sala da L4, ele conseguiu percorrer a pista porém em alguns momentos, principalmente próximo as paredes brancas, ele precisou de um auxílio para retornar a pista. No geral, o Jaguar conseguiu percorrer a primeira pista e boa parte da segunda, mostrando assim um resultado bem satisfatório para o projeto.
Todos os arquivos do projeto podem ser encontrados no \textit{link} do \textit{GitHub}: https://github.com/PedroUnifor/behavioral-cloning-one-camera, assim como os video no link do \textit{YouTube}:
https://youtu.be/l4X\_hcGPhMY.