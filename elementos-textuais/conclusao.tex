\chapter{Conclusões e Trabalhos Futuros}
\label{chap:conclusoes-e-trabalhos-futuros}

Mesmo diante de algumas limitações, o Jaguar consegue nevegar de forma autônoma. Na primeira pista o seu desempenho foi excelente, o que surpreendeu a todos pelo pouco treinamento da rede neural e escassez de recursos tanto para o teste quanto para o treinamento.

Na segunda pista o seu desempenho foi satisfatório, porém não tão bom quanto na pista de atletismo na Unifor. Em alguns momentos ele agia como se não conseguisse visualizar as bordas da pista, acabando fugindo um pouco dela. Porém, por mais que alguns momentos tivesse que parar para ajustá-lo, ele conseguiu realizar todas as curvas do percurso, satisfazendo assim as exigências básicas para a conclusão do projeto.

\section{Limitações}
\label{sec:limitacoes}

Sem dúvida a maior limitação foi a perda da bateria do Jaguar. Sem ela, foi preciso usar cabos para ligar o Jaguar diretamente na tomada, impossibilitando dele percorrer longas distâncias na quadra de atletismo na Unifor e fazer um longo percusso na sala da L4 por causa das pilastras.

Para fazer um bom treinamento é preciso ter uma grande quantidade de fotos. Algoritmos de Aprendizagem Profunda foram desenvolvidos para trabalhar com uma imensa quantidade de dados e, no caso de imagens, eles chegam a trabalhar com mais de dez mil imagens. Nesse projeto, cada treinamento foi feito com no máximo duas mil imagens devido as limitações de cabo e tempo, resultando em modelos não tão otimizados.

O material para fazer a segunda pista era o de pior qualidade. Toda vez que o Jaguar confundia uma trajetória ele destruía parte da pista e mesmo consertando ela não ficava tão perfeita quanto era antes. Isso atrapalha na hora da criação de novos modelos e até para os testes.

As cores das paredes e pistas provavelmente causaram uma confusão no algoritmo da rede neural. As bordas das pistas eram brancas, assim como as paredes da sala da L4 e o chão da pista era o mesmo chão de fora da pista. O algoritmo precisa de cores diferenciadas para saber por onde o deve percorrer. Se as cores são parecidas, provavelmente ele não vai diferenciar com clareza o que é e o que não é pista.

\section{Trabalhos Futuros}
\label{sec:trabalhos-futuros}

Esse trabalho foi apenas o pontapé inicial de um projeto que ainda pode crescer muito. Se o Jaguar for levado para outras pistas na intenção de obter novos dados de treinamento será possível aperfeiçoar ainda mais o seu modelo de navegação autônoma. Além disso, o jaguar possui sensores como GPS  (sistema de posicionamento global) e \textit{LiDAR} (sensor que mede a distância de objetos próximos) que podem ser utilizados para trabalharem juntos com a câmera e o algoritmo de inteligência artificial aperfeiçoando ainda mais a navegação autônoma da plataforma.