\chapter{Conclusões e Trabalhos Futuros}
\label{chap:conclusoes-e-trabalhos-futuros}

Mesmo diante de algumas limitações, o Jaguar consegui sim ser pilotado de forma autônoma. Na primeira pista o seu desempenho foi excelente, o que surpreendeu a todos pelo pouco treinamento da rede neural a escassez de recursos Tanto para o teste quanto para o treinamento.

Na segunda pista o seu desempenho foi satisfatório, porém não tão bom quanto na pista de atletismo na Unifor. Em alguns momentos ele agia como se não conseguisse visualizar as bordas da pista, acabando que fugindo um pouco dela. Porém, por mais que alguns momentos tivesse que parar para ajusta-lo, ele conseguiu realizar todas as curvas do percurso em ambas as direções, satisfazendo assim as exigências básicas para a conclusão do projeto.
\section{Contribuições do Trabalho}
\label{sec:contribuicoes-do-trabalho}

\section{Limitações}
\label{sec:limitacoes}

Sem duvida a maior limitação foi a perda da bateria do Jaguar. Sem ela, foi preciso usar uma afiação para ligar o Jaguar diretamente na tomada, impossibilitando dele percorrer longas distancias na quadra de atletismo na Unifor e fazer um longo percusso na sala da L4 por causa das pilastras. 

Outro grande limitante foi a falta de uma boa câmera e de imagens. Para fazer um bom treinamento é preciso ter uma grande quantidade de fotos. Algoritmos de Aprendizagem Profunda foram desenvolvidos para trabalhar com uma imensa quantidade de dados e, no caso de imagens, eles chegam a trabalhar com milhões de imagens. Nesse projeto, cada treinamento foi feito com no máximo duas mil imagens e de baixa qualidade, já que a câmera do Jaguar não tão boa, dificultando ainda mais o treinamento e os testes.

O material para fazer a segunda pista era o de pior qualidade. Toda vez que o Jaguar confundia uma trajetória ele destruía parte da pista e, mesmo consertando ela, não ficava tão perfeito quanto era antes. Isso atrapalha na hora da criação de novos modelos e até para os testes.

\section{Trabalhos Futuros}
\label{sec:trabalhos-futuros}





