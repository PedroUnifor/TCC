
Cavalcante, Pedro Benevides. Navegação autônoma da Plataforma Jaguar utilizando Rede Neurais Convolucional. Trabalho de Conclusão de Curso (Graduação) – Engenharia de Controle e Automação. Universidade de Fortaleza - Unifor 2018.   

O presente trabalho tem como objetivo adaptar o algoritmo do Nanodegree da Udacity de Engenheiro de carro Autônomo para o para a plataforma robótica Jaguar, um robô pertencente a Unifor que possui apenas uma câmera, diferentemente do projeto usado como base, que possui três câmeras. 
O Jaguar será controlado por um computador conectado a ele através de uma rede \textit{Wireless} onde o \textit{ROS}, que uma plataforma com ferramentas e bibliotecas para desenvolvimento em robótica, possa operar operar e se comunicar com a plataforma. O computador fará a predição dos movimentos do Jaguar analisando as imagens capturadas pela câmera por um algoritmo de Rede Neural com uma camada de convolução. Esse algoritmo é programado em python e utiliza a biblioteca \textit{Keras} para o desenvolvimento da Inteligencia Artificial que vai fazer a predição. Com redes neurais artificiais e uma camada de convolucional é possível fazer  com que o Jaguar seja controlado de forma totalmente autônoma.

\textbf{Palavras-chave}: \textit{Ros}, \textit{Keras}, \textit{Deep learning}, \textit{TensorFlow}, Inteligencia Artificial, Rede Neural Artificial, Rede Neural Convolucional

