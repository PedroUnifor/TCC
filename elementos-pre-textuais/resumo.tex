
O presente trabalho tem como objetivo adaptar o algoritmo do \textit{Nanodegree} da \textbf{Udacity} de Engenheiro de carro Autônomo para a plataforma robótica Jaguar, um robô pertencente a Unifor que possui apenas uma câmera, diferentemente do projeto usado como base, que possui três câmeras.
O Jaguar é controlado por um computador conectado a uma mesma rede \textit{Wireless} gerenciada pelo ROS. O computador receberá a imagens fornecidas pela câmera do Jaguar juntamente com dados de angulo de rotação e aceleração no exato instante de captura da imagem. Com esses dados, a rede neural irá criar um modelo de predição para fazer a navegação autônoma do Jaguar. Todo esse processo é realizado no sistema operacional Ubuntu e é utilizado o python para a criação dos programas e da rede neural com camadas convolucionais.
%O Jaguar será controlado por um computador conectado a ele através de uma rede \textit{Wireless} onde o \textit{ROS}, que uma plataforma com ferramentas e bibliotecas para desenvolvimento em robótica, possa operar operar e se comunicar com a plataforma. O computador fará a predição dos movimentos do Jaguar analisando as imagens capturadas pela câmera por um algoritmo de Rede Neural com uma camada de convolução. Esse algoritmo é programado em python e utiliza a biblioteca \textit{Keras} para o desenvolvimento da Inteligência Artificial que vai fazer a predição. Com redes neurais artificiais e uma camada de convolucional é possível fazer com que o Jaguar seja controlado de forma totalmente autônoma.

\textbf{Palavras-chave}: \textit{Ros}, \textit{Keras}, \textit{Deep learning}, \textit{TensorFlow}, Inteligencia Artificial, Rede Neural Artificial, Rede Neural Convolucional
