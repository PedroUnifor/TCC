Cavalcante, Pedro Benevides. Autonomous navigation of the Jaguar Platform using Convolutional Neural Network. Course Completion Work (Graduation) - Control Engineering and Automation. University of Fortaleza - Unifor 2018.

The present work aims to adapt the algorithm of Nanodegree from Udacity of Autonomous Car Engineer to the one for the Jaguar robotic platform, a robot belonging to Unifor that has only one camera, unlike the design used as base, which has three cameras.
Jaguar will be controlled by a computer connected to it through a wireless network where ROS, which a platform with tools and libraries for development in robotics, can operate to operate and communicate with the platform. The computer will predict the movements of the Jaguar by analyzing the images captured by the camera by a Neural Network algorithm with a convolution layer. This algorithm is programmed in python and uses the {Keras} library for the development of Artificial Intelligence that will do the prediction. With artificial neural networks and a layer of convolutional it is possible to make the Jaguar be controlled totally autonomously.


\textbf{Keywords:} {Keras}, {Deep learning}, {TensorFlow}, Artificial Intelligence, Artificial Neural Network, Convolutional Neural Network